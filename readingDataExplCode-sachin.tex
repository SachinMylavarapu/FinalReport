\subsection{Reading and Conditioning Data in MATLAB (Sachin)}
\label{sec:RandCDatainMATLAB}

The code used for the training and testing of the data can be found in Appendix \ref{ch:HMMCode} for ease of explanation. Firstly, we examine the main script, covering the reading and conditioning of data, spectrogram transformation and PCA analysis. After explaining the various functions created and used in the script till then, we move on to cover the training and testing of the data.

From the code found in the Appendix \ref{sec:apneaHMM}, the \verb!trainIndex! and \verb!testIndex! vectors are used simply for selecting the files to be used, out of 35, for training and the remainder (or less) for testing the accuracy of the diagnosis. Having chosen the files, the next step is to extract and read the files. This is done using the \verb!readData! function, presented in Appendix \ref{ch:readingDataCode}.

The function reads data from the indices specified in \verb!fileIndex! (which is trainIndex in the main script above), and returns O, containing all the observations merged together in a TxD matrix. T is the total number of minutes of data, and D is the number of samples in a minute (6000 in this case), such that there are T annotations in total. The function also returns the vector q, a Tx1 vector containing the latent states for every minute in O, as well as consolidated time, signal and annotation time vectors for ease of plotting and analysis later on.

Firstly, \verb!readData! uses a simple function \verb!getFilenames! to return a 35x1 cell of the available filenames, in a cell string. Then, after initialising the variables, \verb!readData! uses a for-loop to run through each file and extract the relevant information. Using the pre-provided \verb!rdsamp! and \verb!rdann! functions, the signal values as well as annotations are read from the file. As the annotations use `A' for apnoeatic episodes and `N' for non-apnoeatic episodes, the vector type is converted to the alphabet ${0,1}$. The \verb!O! and \verb!q! output matrices are built up using the information from each file, and finally some trivial conditioning is done to ensure ease of plotting if the signal were to be kept.

Armed with the consolidated vectors \verb!X! and \verb!Y!, we now proceed to use the \verb!spectrogram! function in \verb!MATLAB!\textsuperscript{\textregistered}, as described in section~\ref{sec:conditioningExperiments-ta}. We then use the \verb!pca! function from the \verb!pmtk3! package to perform Principal Components Analysis (choosing the number of principal components we wish to include, \verb!k!). 
\chapter{Code for the simple model (Sachin)}
\label{ch:simpleCode}

\section{Main script -- simple.m}
\label{sec:simple}
\begin{lstlisting}
% Reading data
clear all;
close all;
FILENAME = 'signal.wav';
[YRaw,f] = audioread(FILENAME) ;
TMAX = length(YRaw) / f;

% Sampling data because we don't need very high resolution
scale = 20;
for i = 1:(length(YRaw) / scale)
    Y(i) = 0.5 * (YRaw(i * scale, 1) + YRaw(i * scale, 2));
end

% Plots sampled data
figure;
plot(Y);
title('Sleep signal');

% Calculates and plots frequency
N = length(Y);
Yf = fft(Y,N);
freq = ((0:1/N:1-1/N)*f).'; % Frequency vector;

% Plot spectrum.
plot(freq,abs(Yf)) 
title('Amplitude Spectrum of y(t)')
xlabel('Frequency (Hz)')
ylabel('|Y(f)|')

% Calculates and plots the power
power = Y.^2;

% Apnea detection
sensitivityMean = 0.01;
interval = 3;
sensitivityVar = 5e-2;
windowSize = 20;
apnea = detectApnea(power, TMAX, sensitivityMean, interval);
apneaVar = detectApneaVar(power , TMAX , sensitivityVar , windowSize);
\end{lstlisting}

\section{Detecting apnoea -- detectApnea.m}
\label{sec:detectApnea}
\begin{lstlisting}
function apnea = detectApnea(Y, TMAX, sensitivityMean, interval)
    n = length(Y); % No. of samples
    sampleInterval = round(n / TMAX * interval); % Minimum no. of consecutive samples that should be below threshold for apnea
    apnea = zeros(1, n); % Generate a vector that will contain either zeros or ones to show where apnea is
    
    threshold = mean(Y) / sensitivityMean;
    
    nBelowThreshold = 0; % Number of sample points below the threshold in a row.
	for i = 1:n
        if Y(i) <= threshold
            nBelowThreshold = nBelowThreshold + 1; % Increase # of sample points below the threshold in a row
        else % We see a point that is above threshold.
            if nBelowThreshold >= sampleInterval % Condition for which we classify apnea.
                for j = max((i - nBelowThreshold), 1):(i - 1) % Loop through the last 'nBelowThreshold' points. Making sure (i - nBelowThreshold) doesn't go below 1 (otherwise error).
                    apnea(j) = 1;
                end
            else
                for j = max((i - nBelowThreshold), 1):(i - 1)
                    apnea(j) = 0;
                end
            end
            nBelowThreshold = 0;
        end
    end
end
\end{lstlisting}

\section{Detecting apnoea using the `variance' method -- detectApneaVar.m}
\label{sec:detectApneaVar}
\begin{lstlisting}
function apnea = detectApneaVar (Y , TMAX , sensitivityVar , windowSize)
n = length (Y);
apnea = zeros (1 , n);

% Calculate variance of whole signal (as a yardstick)
variance = var(Y) ;
maxVariance = variance / sensitivityVar;

% Calculate the variance through a moving window
twindow = [1:windowSize ; Y( 1:windowSize )]; % This vector contains the locations of the moving window

for i = 1:(n - windowSize)
    sigma2 = var( twindow(2, :) );
    
    if sigma2 > maxVariance 
        apnea(1 , twindow(1,:) ) = 1;
    end
    
    twindow(1,:) = twindow(1,:) + 1; % Update the window for the next round of variance calculations
    twindow(2,:) = Y ( twindow(1,:) );
end
\end{lstlisting}
\chapter{Code for data conditioning}
\section{Transforming to spectrogram space -- transformSpectrogram.m}
\label{sec:transformSpectrogram}
\begin{lstlisting}
function [OTransformed, qTransformed] = transformSpectrogram(O, q)
% O = matrix of T D-dimensional observations [TxD]
% q = vector of T hidden states [Tx1]
% OTransformed = spectrograms after transformation and cut-off [T'xD']
% qTransformed = hidden states after cut-off [T'x1]

    [T, D] = size(O);

    freq = 100;
    nBinsPerAnn = 6; % number of bins per annotation
    window = D / 2; % length of window
    nOverlap = window - D / nBinsPerAnn;
    OBig = reshape(O', 1, T * D);
    [Ss, Fs, Ts, Ps] = spectrogram(OBig, window, nOverlap, [], freq); % time on x axis, freq on y axis
    
    timeCutOff = (T - 1) * nBinsPerAnn;
    freqCutOff = 1500;
    Ps = Ps(1:freqCutOff, 1:timeCutOff);

    OTransformed = reshape(Ps, nBinsPerAnn * size(Ps, 1), T - 1)';
    qTransformed = q(1:(T - 1));
end
\end{lstlisting}

\section{Calculating PCA parameters -- pcaCalc.m}
\label{sec:pcaCalc}
\begin{lstlisting}
function [PCoef, PVec, xMean, xVar] = pcaCalc(x, k)
% Calculates the PCA parameters to reduce D-dimensional vectors stored in x
% [NxD] to K-dimensional vectors stored in y [NxK]
%   PCoef = vector of eigenvalues (variances)
%   PVec = k principal bases stored in k columns [DxK]
%   xMean = mean of data x (for future normalisation purposes)
%   xVar = var of data x (for future normalisation purposes)
    
    % Normalise
    xMean = mean(x);
    xVar = var(x, 1);
    x = bsxfun(@minus, x, xMean);
    x = bsxfun(@rdivide, x, sqrt(xVar));
    
    % PCA
    [PCoef, PVec] = pca(x); % DxD, with i-th column being the i-th principal component (from PMTK3 package)
    PVec = PVec(:, 1:k); % consider only the first k components
end
\end{lstlisting}

\section{Reducing dimensions using PCA -- pcaTransform.m}
\label{sec:pcaTransform}
\begin{lstlisting}
function y = pcaTransform(x, PVec, xMean, xVar)
% Reduces dimensions of D-dimensional vectors in x [NxD] to K-dimensional
% vectors stored in y [NxK] using PCA parameters

    % Normalise
    x = bsxfun(@minus, x, xMean);
    x = bsxfun(@rdivide, x, sqrt(xVar));
    
    % PCA
    y = x * PVec;
end
\end{lstlisting}
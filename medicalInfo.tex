\chapter{Medical Information}
\label{ch:medicalInfo}
Obstructive Sleep Apnea (OSA), sometimes called hypersomnia sleep apnea syndrome, occlusive sleep apnea, sleep disordered breathing, hyperventilation syndrome and Pickwickian Syndrome, however the latter three are discouraged because they are also used to describe other disorders, is a sleep disorder characterised by repetitive blockages in the upper airway during sleep, but with inspirational effort. These blockages can be apneas, full closure, or hypopneas, partial closure, both cause a restriction in airflow, which can lead to reduction in oxygen saturation and frequent arousal in order to re-establish airflow. 

The upper airways include the nasal cavity, oral cavity and pharynx, the area behind the tongue, the main part that becomes blocked is the pharynx, this is usually held open by the pharyngeal dilator muscles, although these relax during sleep, in a healthy subject they are still able to maintain airflow, however in an OSA sufferer they provide insufficient force to prevent collapse. Collapse occurs during inspiration only due to negative pharyngeal pressure. During rapid eye movement (REM) sleep there is further relaxation of the pharyngeal dilator muscle manifesting as a decrease in tone and activity which can lead to longer apnea and hypopnea events. 

Many factors affect the likelihood a patient suffers with OSA, however one thing is consistent in almost all cases; the pharyngeal upper airway size is smaller than in normal patients and often more elliptical, this can have a number of causes including fat deposits and facial bone structure. Overweight and obese patients often have fat deposits lateral to the pharynx, not always substantial but its positioning creates or reinforces elliptical shape and reduction in size. The main element of facial bone structure that influence upper airway size is the positioning of the maxilla and mandible ( upper and lower jaw bones), these can influence the airway size in a number of ways, retrognathia where the mandible is smaller and therefore appears to be set back or both jaw bones can be the same size but set back in both of these cases the tongue sits further back in the mouth, increasing tendency to block airflow. Lower positioning of the hyoid bone and Brachycephaly where the head is wider than it is tall can have a similar effect. 

Abnormal facial tissues can also have an effect, large tonsils and adenoids, elongated or enlarged uvula (the dangly bit at the back of the mouth), macroglossia (enlarged tongue), high arched or narrow hard palate, reduced nasal patency (cross sectional area) possibly caused by nasal abnormalities all have been shown to factor.

Other factors include lying supine (on back) which allows gravity to pull the tongue into the airway. Some drugs including alcohol relax the pharyngeal dilator muscles more than just the effects of sleep, worsening OSA. Smoke irritates tissues including those in the upper airways, swelling them such that they cause a narrowing of the airway.
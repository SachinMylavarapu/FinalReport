\chapter{Introduction (George)}
\label{ch:introduction}
% The division of labour among the group members is outlined in \fullref{ch:divisionOfLabour}.

Obstructive Sleep Apnea is a highly prevalent condition worldwide and is understood to have a cause link with several factors, both biological and environmental. Diagnosis is currently attained after the patient visits a sleep laboratory; this process is costly both monetarily and in terms of the time required by specialists to carry out the full diagnosis. Currently the preliminary diagnosis stage before this is by way of a home testing kit, but again this has an associated cost and requires some training before use.
This project explores the use of a smartphone app to provide a primary form of diagnosis, such that a General Practitioner can be better informed as to whether a patient needs further tests and/or treatment. The project aims to remove the need for specialist knowledge to carry out the diagnosis and the clearest way to do this is by a non-invasive technique. As the condition is characterised by distinctive snoring sounds, the focus has been on using audio for data input. The project is constrained by the commonly available hardware and software capabilities of smartphones. It looks to maximise the accuracy of the diagnosis using some advanced techniques of data handling, as well as discuss some of the detail pertaining to the physical implementation of a smartphone app.
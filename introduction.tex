\chapter{Introduction (George)}
\label{ch:introduction}
% The division of labour among the group members is outlined in \fullref{ch:divisionOfLabour}.

Obstructive Sleep Apnea (OSA) is a highly prevalent condition worldwide and is understood to be linked with several causing factors, both biological and environmental. Diagnosis is currently attained after the patient visits a sleep laboratory; this process is costly both monetarily and in terms of the time required by specialists to carry out the full diagnosis. The main alternative diagnosis to this is by way of a home testing kit, but again this has an associated cost and requires some training before use.

This project explores the use of a smartphone app to provide a preliminary form of diagnosis which can then be provided to a General Practitioner to recommend whether a patient needs further tests and/or treatment. The project aims to remove the need for specialist knowledge to carry out the diagnosis and the clearest way to do this is by a non-intrusive technique. As the condition is characterised by distinctive snoring sounds, audio has been identified as a useful input to characterise OSA, along with the simple but effect method of a questionnaire to identify the disposition to known causal factors. The project is constrained by the commonly available hardware and software capabilities of smartphones. It looks to maximise the accuracy of the diagnosis using some advanced techniques of data handling, as well as discuss some of the detail pertaining to the physical implementation of a smartphone app.
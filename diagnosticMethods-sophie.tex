\section{Diagnostic Approaches (Sophie)}
\label{sec:diagnosticMethods-sophie}
Before it is possible to work out how to design the app, the overall mechanism of diagnosis of the app needs to be decided, as well as which approach to take in order to work out which signals and sensors to use for the app.

Diagnosis can be approached in a number of ways depending on whether the Doctor (or algorithm) thinks they recognise the condition or there is a standard flow chart for diagnosis, amongst other things. These approaches have been formalised and fall into four categories, they are: the algorithmic method; pattern recognition method; hypothetico-deductive method, sometimes called differential diagnosis; and the exhaustive method ~\cite{ebpdiagnosticstrategies,mengel2002fundamentals}.

The algorithmic method uses flowcharts and algorithms to analyse data that are precise and reproducible, for example vitamin B12 level in blood. One follows steps making decisions at preselected branch point based on the clinical data available. This relies on a flow chart for the condition and for the condition to present in a relatively normal way. Abnormal presentation of a condition could easily lead to misdiagnosis. For example if an OSA sufferer is not overweight but instead has a high arched hard palate, they may well be missed by this method. 

The pattern recognition method is best for conditions with distinct presentation, especially those which present regularly in the population. It is an efficient method, especially useful when time critical diagnoses are needed. This does however risk jumping prematurely to a final hypothesis, although the consequences of this can be reduced by follow up in order to check the initial instincts. 

The hypothetico-deductive method lends itself to primary care settings as it results in a differential diagnosis, a rank ordered list of hypotheses. Hypotheses are generated and rejected as more data are collected and questions asked until a working hypothesis is reached. This method is most often used as it reflects how most people deal with life, and is therefore the most natural. It can cause errors if a doctor has been exposed to a certain diagnosis recently, biasing their importance weightings of certain symptoms. For example if a doctor has diagnosed a number of patients with OSA recently (or read a paper about it) they may diagnose a patient who expresses day time sleepiness and witnessed apnoeas with OSA when in fact the previous traumatic head injury would be an indicator for Central Sleep Apnoea. In this case a sleep test would be needed to see whether the patient was attempting respiration during apnoeas. Another error situation can be caused by the doctor failing to think about the probability of a diagnosis being correct. For example if a patient presents with witnessed choking during sleep as the primary symptom, rather than daytime sleepiness or witnessed apnoeas, the doctor may diagnose Sleep Choking Syndrome. However this has a far lower prevalence than OSA; it is considered rare by AASM ~\cite{american2001international}.

The exhaustive method works on the premise of having all the information, all data collected, all questions asked. Completeness of data is important for hospitalised patients but acquiring it is too time consuming and expensive for most cases. It is useful for unusual conditions when other methods have failed or for unusual expressions of conditions. 

The hypothetico-deductive method will be useful in order to establish what symptoms and signs can be used to differentiate OSA from other disorders. However the app itself will need to rely on pattern recognition, using standard and obscure symptoms of OSA to confirm diagnosis and rule out other options (or not). It would be unnecessary to use an exhaustive method as the app is only designed to pick up OSA not all sleep disorders or reasons for daytime tiredness.

\section{Diagnostic Methods}
\label{sec:diagnosticMethods-sophie}
There are four strategies used to approach diagnosis they are; the algorithmic method, pattern recognition method, hypothetico-deductive method; sometimes called differential diagnosis, and the exhaustive method. 

The algorithmic method uses flowcharts and algorithms to analyse data that are precise and reproducible for example vitamin B12 level in blood. One follows steps making decisions at preselected branch point based on the clinical data available. This relies on a flow chart for the illness and the illness to present in a relatively normal way. Abnormal presentation of an illness could easily lead to misdiagnosis. For example if an OSA sufferer is not overweight but instead has lateral peritonsular narrowing, they may well be missed by this method. 

The pattern recognition method, is best for conditions with distinct presentation, especially those which present regularly in the population. It is refered to as pattern recognition because the symptoms and signs displayed by the patient reawakens memories in the doctor of previous examples. It is an efficient method, especially useful when time critical diagnoses are needed. This does however risk jumping prematurely to a final hypothesis, although the consequences of this can be reduced by follow up in order to check instincts. 

The hypothetico-deductive method lends itself to primary care settings as it results in a differential diagnosis, a rank ordered list of hypotheses. Hypotheses are generated and rejected as more data are collected and questions asked until a working hypothesis is reached. This method is most often used as it reflects how most people deal with life, and is therefore the most natural. It can cause errors if a doctor has been exposed to a certain diagnosis recently, they may choose to ignore signs that what they are looking at is in fact not that. For example if a doctor has diagnosed and number of patients with OSA recently ( or read a paper about it) they may diagnose a patient who expresses day time sleepiness and witnessed apneas with OSA when in fact the previous traumatic head injury would be an indicator for Central Sleep Apnea, in this case a polysomnogram would be needed to see whether the patient was actually attempting respiration during apneas. Another error situation can be caused by the doctor failing to think about the probability of a diagnosis being correct. for example, if a patient presents with witnessed choking during sleep as the primary symptom rather than daytime sleepiness or witnessed apneas the doctor may diagnose Sleep Choking Syndrome however this has a much lower prevalence than OSA, rare compared to about 2-4\% for OSA.

The exhaustive method works on the premise of having all the information, all data collected, all questions asked. Completeness of data is important for hospitalised patients but acquiring it is too time consuming and expensive for most cases. Useful for unusual illness when other methods have failed or for unusual expressions of illnesses. 

The hypothetico-deductive method will be useful in order to establish what symptoms and signs can be used to differentiate OSA from other disorders. However the app itself will need to rely on pattern recognition because ruling out all options will be impossible as a doctor won’t be administering the app. It was also be unnecessary to use an exhaustive method as the app is only designed to pick up OSA not all sleep disorders or reasons for daytime tiredness. 
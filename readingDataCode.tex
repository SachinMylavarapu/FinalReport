\chapter{Code for reading data -- readData.m}
\label{ch:readData}
\begin{lstlisting}
function [O, q, time, signal, annTime] = readData(fileIndex, keepSignal)
% Reads data from indices specified in the vector fileIndex and returns
%   O = observations, merged together [TxD]
%   q = latent states, 1-0 for apnea-nonapnea [Tx1]
% Inputs
%   fileIndex = vector of file indices to read
%   keepSignal = keep signal and time for plotting purposes (set to 1 or 0)
    
    disp('Reading and conditioning data...');
    filenames = getFilenames();
    
    lastTime = 0;
    annTimeLast = 0;
    time = []; % time
    signal = []; % signal
    annTime = []; % timestamps of annotations
    q = []; % latent states
    O = []; % observations
    for i = fileIndex
        filename = cell2mat(filenames(i));
        disp(sprintf('\tProcessing file %d: %s...', i, filename));
        
        [timeTemp, signalTemp, freq] = rdsamp(filename); % reads the signal
        [annTimeTemp, type] = rdann(filename,'apn'); % reads annotations
        type = (type == 'A');

        %% Conditioning data
        annTimeTemp = [1; annTimeTemp];
        q = [q; type]; % latent states
        for i = 1:length(type)
            O = [O; signalTemp((annTimeTemp(i) + 1):annTimeTemp(i + 1))'];
        end
        
        if keepSignal
            nObs = annTimeTemp(end);
            time = [time; timeTemp(1:nObs) + lastTime + 0.01];
            annTime = [annTime; annTimeTemp(2:end) + annTimeLast];
            signal = [signal; signalTemp(1:nObs)];
            lastTime = time(end);
            annTimeLast = annTime(end);
        end
    end
    if ~keepSignal
        signal = [];
        time = [];
    end
end
\end{lstlisting}
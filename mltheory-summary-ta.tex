\subsection{Summary}
	We have explored the theory behind three models: Support Vector Machines (SVMs), State-Space Models (SSMs), and Hidden Markov Models (HMMs). While the SVMs model data discriminatively, the SSMs and HMMs do it generatively. In our project, we will use SVMs and HMMs to learn from actual sleep data and use the learnt model to classify unseen test data. We compare the performances of both methods by comparing the proportion of correct classifications.

	We have also discussed two methods to condition our data: using spectrograms and Principal Components Analysis (PCA). In our experiments, we will transform our data into spectrogram space and then perform PCA to choose only a few principal components to reduce the dimensionality of our data.

	The model with the best performance, together with the conditioning methods will be chosen for implementation in Java for Android integration.
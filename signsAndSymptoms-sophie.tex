\section{Signs and Symptoms -- non-invasive methods}
\label{sec:signsAndSymptoms-sophie}
Table X shows all the known signs and symptoms of OSA along with detectable presentation. From there Table X+1 shows these presentations and the different sensors for measuring them, as well as whether those sensors are invasive or not. The non-invasive sensors shall be the focus of the following work. The presenting symptom for OSA is daytime sleepiness, with reports of snoring and witnessed apneas from the patient’s bed partner. So focussing on other conditions with these symptoms is a start to finding a unique combination of symptoms for OSA that can be tested non-invasively by an app. 

Table X+2 looks at four key symptoms of OSA which the majority of patients display ( and can be detected non-invasively). Disorders which also display one or more of these symptoms are then listed along with their differentiator from OSA. From the table it is clear most of these conditions can be ruled out by diagnosis focussing on daytime sleepiness, snoring, and an apnea awakening/choking combination. 

Ascertaining these would eliminate all but central sleep apnea and altitude insomnia however both of these would be picked up a significant proportion of the time by a questionnaire because the sufferers would only fit the risk cases for OSA in a small percentage of cases. Especially for altitude insomnia as those with a BMI over 35kgm-2 are unlikely to be climbing over 4000m in height for extended periods ( i.e. climbing mountains rather than flying).

Those suffering from central sleep apnea are likely to reflect population distribution for BMI as weight is not a cause or exacerbating factor. I.e. about 25\% of central sleep apnea patients are obese ($\text{BMI} > 30$) compared to 70\% of OSA sufferers.

Methods are then needed to detect daytime sleepiness, loud snoring and apnea arousal (choking) combinations. Sensors to look at are; video, accelerometry, audio and questionnaires. Idea and their merits and weaknesses will be discussed next.

\subsection{Video}
The aim is to pick up on apneas via cessation of breathing and arousal via significant body movement. Snoring would not be possible to detect although some other attributes that hint at OSA may be detectable e.g. lying in the supine position ( on back). Video could also be used to detect sleep and activity, e.g. sleep cycle transitions and limb movements. 

Merits include, smartphone already have cameras so an external camera wouldn’t be needed, although the cameras may struggle with the low light level in the sleeping space, and changing the light level in order to accommodate the camera could have a knock on effect on the quality or type of sleep. 

Weaknesses include, positioning of the camera will be important and this is hard when all rooms are different and smartphones don’t naturally stand up by themselves. Manual analysis of the video has been used until recently so there isn’t a great deal of statistical data on how well computer algorithms can detect OSA. Video produces large data files that cannot be sensibly stored on a phone. Video has formed part of the polysomnogram in some areas for a long time as an aide to observation and only recently investigated as an independent method of diagnosis using automated analysis. As a consequence lots of video data does exist however it is highly regulated due to patient confidentiality. 

\subsection{Accelerometry}
This has the potential to be able to detect body position and sleep wake segmentation which could give a pretty good indicator of apnea arousal events, however mechanism for detecting snoring.

This has not been part of the polysomnogram although it has started to be investigated as a mechanism for detecting OSA. There is very little data available which makes a retrospective study a challenge, especially as there is no data conbined with polysomnogram data for verification. Positioning of the phone in order to get the best results without damaging the phone would require some investigation. Non standardised arrangements of mattress and bed clothes may cause issues. 

\subsection{Audio}
The most obvious way to detect audio signals from the patient is via the phone’s in built microphone although external microphones should also be investigated. It ought to be possible to detect apnea arousal events and snoring. 

Lots of work has been done on audio analysis of speech so there are lots of ideas out there to work from. Many polysomnograms include a sound recording for reference so there is a lot of data available and in the most part it is anonymised so there isn’t a risk to patient confidentiality. In some cases it has been labelled so it can be used easily for a retrospective study such as this one. There are also regulations which can act as a guidline. 

\subsection{Questionnaire}
A questionnaire is pretty much the only way to assess daytime sleepiness, it can also be used to acquire about other symptoms and co-morbidities. If a prexisting questionnaire is used then there will be data on its effectiveness and usability and statistics it has produced. Questions for part of the diagnostic pathway so it takes some of the work away from the doctor. Some studies show patients are more truthful with anonymous questionnaires than with their doctor, more on that to follow.

\subsection{Conclusion}
Given that audio analysis and questionnaires should be sufficient to diagnose OSA and rule out other similar disorders, these will be investigated further with the other methods discarded unless significant issues are encountered with audio and questionnaire analysis.
\section{Questionnaire selection}
\label{sec:questionnaire-sophie}
The key point that needs to be established via the questionnaire is whether the patient suffers from daytime sleepiness however knowing about the patient’s weight can be useful to eliminate central sleep apnea, and other questions about symptoms can be useful to reinforce diagnosis. 

\subsection{Own vs Pre-Existing}
There is a key decision between two options at this stage, to create a new questionnaire from scratch or to use a pre-existing questionnaire. A new questionnaire gives much more freedom as to what questions to ask in order to target particular symptoms and eliminate other disorders. It also allows for unique methods of analysis e.g. weighted questions rather than a simple threshold based on number of questions answered. Rigorous market research can be used to design the wording of the questions to maximise truthfulness and ease of interpretation, as well as number of questions and mechanism of answering. However this would be time consuming and an expert is question production is potentially needed. Data from patients would need to be collected via the questionnaire and other means in order to test its effectiveness. 
Using pre-existing questionnaire allows for quicker set up time as there will either be data available in order to validate the questionnaire or it will have already been validated. Some questionnaires are within the doctor’s guideline so time can be saved during consultation if they are already performed. Named questionnaires are often recognised by doctors which saves them time checking the questions and they are also more likely to trust questionnaire they know. 
Given the time constraints and the access to data and skills further investigation will look at pre-existing questionnaires. 

\subsection{How Truthful are People}
One of the benefits of performing questionnaires on the app rather than waiting for patients to attend an appointment is saving doctors time however there is another benefit in the form of increasing truthfulness. 
One study found that patients tended to lie in order to look good, in the context of sexuality, those involved had a strong interest in not giving a disappointing impression. 
Another study looked at doctors’ trust in patients which is an indicator of suspected lying. Ulterior motives were thought to have a significant influence on patients’ truthfulness. In the case of OSA there is a risk of patients playing down daytime sleepiness in order to avoid diagnosis if they wish to avoid having to contact the DVLA and their car insurance company and risk losing their license or insurance due to misunderstanding. There is also a case that patients may try and get a diagnosis in order to receive disability related financial support, however tricking a night time study into thinking you have OSA would be challenging. 

\subsection{Questionnaires}
There are a number of questionnaires that have been used to assess diagnosis of OSA, some rely on a doctor to measure physical features, others use characteristic clinical features, others use patients’ interpretations of their symptoms, and the rest use combinations of the above. 

\begin{itemize}
\item Viner et al\\
Model incorporates snoring, BMI, age and gender. Used stepwise linear logistic regression.
\item Maislin et al\\
Model incorporates snoring, gasping at night, witnessed apneas, age, gender and BMI. Multiple logistic regressions were used to generate a multivariable apnea risk register when compared to RDIs from polysomnographs. ROCs were used to test predictive ability. 
\item Metzer et al - Berlin Questionnaire\\
Model incorporates snoring, gasping at night, witnessed apneas, age, gender and BMI. Multiple logistic regressions were used to generate a multivariable apnea risk register when compared to RDIs from polysomnographs. ROCs were used to test predictive ability. 
\item Kirby et al – Artificial Neural Nets\\
Uses 23 clinical variables including patient’s history, physical examination and patient reported sleepiness and smoking.
\item Dixon et al\\
Looked at witnessed apneas, neck circumference and BMI. Study rather than a questionnaire. 
\item Kushida et al – Kushida Index\\
Complicated morphometric model, including BMI, neck circumference, palatal height and oral cavity measurements amongst others. Suffers from being too complicated to be administered accurately and being time consuming.
\item Tsai et al - upper airway physical examination protocol (UAPP)\\
Looked at six parameters, three clinical symptoms; snoring, witnessed apneas and hypertension and three measurable signs; cricomental space, pharyngeal grade and overbite ( info on these to follow)
\item Chung et al – STOP questionnaire\\
Four questions on snoring, daytime sleepiness, witnessed apneas and blood pressure.
\item Chung et al –STOPbang questionnaire\\
This is an extension of the STOP questionnaire that adds four additional questions on BMI, age, neck circumference and gender. A threshold of 3 out of 8 is generally used to indicate OSA. 
\item Meoli et al – AASM OSA exploratory questionnaire\\
This questionnaire is referenced in a number of places however there is no data to back it up
\item Chung et al – ASA\\
Three categories are used to ask questions, physical characteristics, observed sleep disturbances and tiredness. It uses falling into two or more of these categories as an indicator of OSA. 
\item Flemons’ et al – Flemons’ screening tool\\
This is a 36 question screening tool for OSA that uses a differential method for diagnosis, asking questions about depression and chronic diseases as well as the more common questions on tiredness, snoring and driving behaviour. Uses a weighted score called SACS with a threshold to indicate OSA. 
\item Johns – Epworth Sleepiness Scale\\
A questionnaire using eight questions to assess sleepiness in different situations. Recommended by NHS guidelines. 
\end{itemize}

\subsection{Narrowing Down the Studies}
A number of these aren’t formal questionnaires and therefore not known by the doctor negating one of the reasons for using a pre-existing questionnaire, however this isn’t sufficient reason to rule them out. The Kushida Index and ASA questionnaire are complicated which goes against the design specification. A high negative predictive value is needed so the focus of more work will be on; Artificial Neural Net, UAPP and the STOPbang questionnaire. Epworth Sleepiness Scale will also be included in the app because it is part of the diagnostic pathway as laid out by the NHS.

\section{Questionnaire selection}
The key point that needs to be established via the questionnaire is whether the patient suffers from daytime sleepiness however knowing about the patient’s weight can be useful to eliminate central sleep apnea, and other questions about symptoms can be useful to reinforce diagnosis. 

\subsection{Own vs Pre-Existing}
There is a key decision between two options at this stage, to create a new questionnaire from scratch or to use a pre-existing questionnaire. A new questionnaire gives much more freedom as to what questions to ask in order to target particular symptoms and eliminate other disorders. It also allows for unique methods of analysis e.g. weighted questions rather than a simple threshold based on number of questions answered. Rigorous market research can be used to design the wording of the questions to maximise truthfulness and ease of interpretation, as well as number of questions and mechanism of answering. However this would be time consuming and an expert is question production is potentially needed. Data from patients would need to be collected via the questionnaire and other means in order to test its effectiveness. 
Using pre-existing questionnaire allows for quicker set up time as there will either be data available in order to validate the questionnaire or it will have already been validated. Some questionnaires are within the doctor’s guideline so time can be saved during consultation if they are already performed. Named questionnaires are often recognised by doctors which saves them time checking the questions and they are also more likely to trust questionnaire they know. 
Given the time constraints and the access to data and skills further investigation will look at pre-existing questionnaires. 

\subsection{How Truthful are People}
One of the benefits of performing questionnaires on the app rather than waiting for patients to attend an appointment is saving doctors time however there is another benefit in the form of increasing truthfulness. 
One study found that patients tended to lie in order to look good, in the context of sexuality, those involved had a strong interest in not giving a disappointing impression. 
Another study looked at doctors’ trust in patients which is an indicator of suspected lying. Ulterior motives were thought to have a significant influence on patients’ truthfulness. In the case of OSA there is a risk of patients playing down daytime sleepiness in order to avoid diagnosis if they wish to avoid having to contact the DVLA and their car insurance company and risk losing their license or insurance due to misunderstanding. There is also a case that patients may try and get a diagnosis in order to receive disability related financial support, however tricking a night time study into thinking you have OSA would be challenging. 

\subsection{Questionnaires}
There are a number of questionnaires that have been used to assess diagnosis of OSA, some rely on a doctor to measure physical features, others use characteristic clinical features, others use patients’ interpretations of their symptoms, and the rest use combinations of the above. 

\begin{itemize}
\item Viner et al\\
Model incorporates snoring, BMI, age and gender. Used stepwise linear logistic regression.
\item Maislin et al\\
Model incorporates snoring, gasping at night, witnessed apneas, age, gender and BMI. Multiple logistic regressions were used to generate a multivariable apnea risk register when compared to RDIs from polysomnographs. ROCs were used to test predictive ability. 
\item Metzer et al - Berlin Questionnaire\\
Model incorporates snoring, gasping at night, witnessed apneas, age, gender and BMI. Multiple logistic regressions were used to generate a multivariable apnea risk register when compared to RDIs from polysomnographs. ROCs were used to test predictive ability. 
\item Kirby et al – Artificial Neural Nets\\
Uses 23 clinical variables including patient’s history, physical examination and patient reported sleepiness and smoking.
\item Dixon et al\\
Looked at witnessed apneas, neck circumference and BMI. Study rather than a questionnaire. 
\item Kushida et al – Kushida Index\\
Complicated morphometric model, including BMI, neck circumference, palatal height and oral cavity measurements amongst others. Suffers from being too complicated to be administered accurately and being time consuming.
\item Tsai et al - upper airway physical examination protocol (UAPP)\\
Looked at six parameters, three clinical symptoms; snoring, witnessed apneas and hypertension and three measurable signs; cricomental space, pharyngeal grade and overbite ( info on these to follow)
\item Chung et al – STOP questionnaire\\
Four questions on snoring, daytime sleepiness, witnessed apneas and blood pressure.
\item Chung et al –STOPbang questionnaire\\
This is an extension of the STOP questionnaire that adds four additional questions on BMI, age, neck circumference and gender. A threshold of 3 out of 8 is generally used to indicate OSA. 
\item Meoli et al – AASM OSA exploratory questionnaire\\
This questionnaire is referenced in a number of places however there is no data to back it up
\item Chung et al – ASA\\
Three categories are used to ask questions, physical characteristics, observed sleep disturbances and tiredness. It uses falling into two or more of these categories as an indicator of OSA. 
\item Flemons’ et al – Flemons’ screening tool\\
This is a 36 question screening tool for OSA that uses a differential method for diagnosis, asking questions about depression and chronic diseases as well as the more common questions on tiredness, snoring and driving behaviour. Uses a weighted score called SACS with a threshold to indicate OSA. 
\item Johns – Epworth Sleepiness Scale\\
A questionnaire using eight questions to assess sleepiness in different situations. Recommended by NHS guidelines. 
\end{itemize}

\subsection{Narrowing Down the Studies}
A number of these aren’t formal questionnaires and therefore not known by the doctor negating one of the reasons for using a pre-existing questionnaire, however this isn’t sufficient reason to rule them out. The Kushida Index and ASA questionnaire are complicated which goes against the design specification. A high negative predictive value is needed so the focus of more work will be on; Artificial Neural Net, UAPP and the STOPbang questionnaire. Epworth Sleepiness Scale will also be included in the app because it is part of the diagnostic pathway as laid out by the NHS.

\section{Audio}

\subsection{Scoring of apnea R\&K vs AASM}
Audio signals taken during polysomnograms are currently scored by sleep specialists, there are two guidelines as to how to do this. The Rechtschaffen and Kales Manual is long standing, it was the only manual in use between 1968 when it was created and 2007 when the AASM Manual came into use. 
The R\&K manual is based on healthy subjects aged 21 to 86 years and doesn’t actually mention how to recognise apneas. It has also been criticised for being open to interpretation. The AASM manual has a slight change in terminology and changes distribution of NREM sleep stages but the main difference is it explains how to classify more sleep abnormalities including apneas and arousals. 
The AASM scores a signal as an apnea if, there is a drop in the peak thermal sensor excursion by more than 90\% of baseline, the event last at least 10 seconds and at least 90\% of the event’s duration meets the amplitude reduction criteria for apnea. Obstructive apnease are associated with continued or increased inspiratory effort through the period of absent airflow. A minimum desaturation criterion is not required. 
The basis for scoring arousals is based on EEG and EMG and therefore isn’t helpful in terms on how to analyse audio signals. 

\subsection{Methods of Analysis}
There are a number of different methods that can be used to analyse audio signals, each will be looked at in turn, looking at how they work, studies on them and strengths and weaknesses of the method. Due to issues with background noise and other signal noise most methods focus on the characteristic snores of OSA sufferers rather than the quiet apneas. 

\subsubsection{Simple Characteristics}
Three simple features of a snore can be analysed quite easily, snore duration, the time taken for a snore sound, generally measured in seconds; snore loudness, generally the average loudness measured from a microphone, a snore simulator is also needed in order to calibrate the microphone; snore periodicity, calculated from segmented short frames of the snores with low amplitude, high frequency components removed, autocorrelation used to asses each frame as either periodic or not. Overall periodicity calculated from the ratio of the number of periodic frames to the total number of frames.

Jones et al studies these features amongst others in order to see whether it was possible to predict the outcome of palatal surgery rather than as a diagnostic measure for OSA. Frame size of 200ms was used centre-clipped by 30\%. Autocorrelation peaks between 25 and 87.5ms were used to classify as periodic. 

\subsubsection{Peak Power}
Using Fast Fourier Transforms a power spectrum can be created of the snores. This can then be characterised in a number of ways including establishing: Fa the fundamental frequency; Fo the lowest frequency, Fpeak the peak with the maximum power, Fmean the statistical mean frequency, and Fmax can also be calculated, however there are different conventions for this, it is defined as the frequency beyond which the signal amplitude has dissipated to less than a percentage of its peak power, some studies use 3\% others 1\%. 
Perez-Padilla et al used Fa, Fo, Fpeak and Fmax to examine the differences between nine OSA sufferers and ten simple snorers. Defining Fmax as dissipating to 3\% of peak power. Sound was recorded via a microphone attached to the manubrium sterni (chest). Significant variation was found between snores in a given patient making it hard to find a differentiator between OSA sufferers and simple snorers. It was found that for OSA suffers Fpeak was usually at a higher frequency than Fa, however this was significant enough to be used to differentiate the two groups.

Fiz et al used Fpeak, Fmean and Fmax defined at 10\% of peak power to distinguish between ten OSA sufferers and seven simple snorers. The microphone was placed just above the larynx without skin contact. Two features were found to distinguish OSA sufferers from simple snorers. The first was peak frequency (Fpeak) which was significantly lower in OSA sufferers with a threshold of 150Hz, although one of each group was on the opposite side, a strong non linear negative correlation was seen between Fpeak value and number of AHI events as seen in figure X ( 3), this is associated with a Spearman rank order correlation: r=-0.70; p<0.0016. The second feature was to do with the shape of the power spectrum. The simple snorers displayed a clear fundamental frequency and harmonic pattern whereas the OSA sufferers displayed a low frequency peak with scattered peaks over a narrow band of frequencies, Figure X (1 \& 2 b) shows the different patterns. Fiz et al attributed the differences seen to microphone placement, Perez-Padilla et al placing their microphone on the chest whereas Fiz et al placing above the larynx, this results in a different filtering effect caused by the different tissues and cavities the sound travels through. 

\subsubsection{Power Ratio}
Given the difference in spectra between those with OSA and simple snorers there is potential to characterise this by means of a cumulative power ratio. This would be the area under one part of the power spectrum divided by the area under the rest, where the threshold(s) is placed would depend on what characteristic was trying to be differentiated. From the power spectrum from the Fiz et al study above one potential threshold would be 100Hz to pick up on the low frequency components of the OSA curve, see figure X ( adapted version of 1\&2 b from Fiz). 
Perez-Padilla et al used superimposed spectra from ten snores from each subject (9 OSA, 10 simple snorers) and a threshold of 800Hz, dividing the integral of the spectra above 800Hz with that below 800Hz, for OSA patients who took a second breath after an apnea the cumulative power ratio was also calculated for that. Figure X (8) shows the scattering of the ratios, with snorers having a ratio of 0.08 $\pm$ 0.02 and OSA sufferers a ratio of 1.12 $\pm$ 0.31. A threshold of 0.3 was proposed which would distinguish all but one OSA sufferer on first breath after apnea, this patient had a low AHI and mild symptoms. 
Hara et al used the same method as above on 46 OSA sufferers and 12 simple snorers, keeping the threshold at 800Hz, using a ratio of below 800Hz to above 800Hz. Simple snorers had a power ratio of 34.002 whereas OSA sufferers had a ratio of 6.288. The p value of a Mann-Whitney U test was 0.015, this small value is a good indicator that this didn’t happen by chance. 
Hara et al wouldn’t recommend this approach as the calculation is very time consuming. Significant differences were found between the values of the ratios in both studies although in both cases a difference between OSA sufferers and simple snorers was clearly seen. 

\subsubsection{Sound Intensity}
Another diagnostic tool uses sound intensity without frequency, instead looking at the apnea-snoring combination. Define an acoustic signature event (ASE) as a period of apnea, duration specified between limits, followed by snoring, where snoring and apnea are defined in terms of sound intensity.
Van Brunt DL et al define apnea below 50microV and snoring above 100microV with apnea episode lasting between 10 and 90 seconds. 30 second intervals were analysed by sound intensity and polysomnogram and analysed in two ways. Firstly each patient (69 patients, 51 OSA suffers, 18 not) was assessed independently and RDI score compared with predictions, for an RDI of 15/hour or greater sensitivity was 0.93, specificity 0.25, false positive 36.2\%, false negative 2.8\%, 60.9\% classified correctly. Secondly pooling all the observations (60231) resulting in a 33\% sensitivity, 98\% specificity, 85\% classified correctly. 

\subsubsection{Formants}
Formants are the resonant frequencies of the signal, most easily determined by picking out the peaks on a linear predictive coding (LPC) spectrum of the signal. LPC is the spectral envelope produced by a linear predictive model. The lowest formants are associated with degree of constriction of the pharynx, degree of advancement of the tongue and degree of lip rounding respectively, these are often referred to as F1, F2 and F3. Because these physical properties change in sufferers of OSA there is a chance that the frequencies associated with them will change too, therefore threshold frequencies to distinguish OSA sufferers and simple snorers are sought. 

Ng et al tested 30 OSA sufferers and ten simple snorers and found a threshold for F1 of 470Hz but not F2 nor F3, this threshold yielded a sensitivity of 88\% and a specificity of 82\%. However increased sensitivity and specificity were seen if different thresholds were used for men and women as seen in table X (1). Women show a greater distinction in F1 frequency than men as well as a reduced spread of results, although more outliers amongst the simple snorers, as seen in figure X (1). It is however worth noting that the sample size for women was small (6 OSA sufferers, 4 simple snorers) Once the threshold has been established an equation is needed to convert it to an AHI value, a number of equations were proposed and regression used to see which was the best fit, as seen in table X(2). A power law came out best with a regression of 0.5334 giving a predicted AHI score of 12.2 when 10 was being aimed for. It is worth noting that the same subjects was used for training and testing, although different data for each.

Sola-Soler et al used formant frequencies to distinguish between snores from eight simple snorers ( 447 snores) and eight OSA sufferers ( 236 normal snores and 429 post-apneic snores). The spectral envelope was estimated by linear predictive autoregression, with very low amplitude spurious peaks rejected by a 3dB threshold. Investigation of the spectral envelope found 2 to 6 formants in each snore, these fell in common frequency range around 150Hz, 500Hz, 1KHz, 1.7KHz, and in a few snores 2.2KHz. This led to definition of five frequency bands: B1[0,300), B2[300,700), B3[700,1400), B4[ 1400,1900) and B5[ 1900,2500) in Hz. For each band and type of snore ( simple snorer SN, OSA normal snore OP-N, and OSA post apneic snore OP-PA) the mean value Fi and standard deviation SFi was calculated. Table X (3) shows a comparison between these means and standard deviations for each combination of snore type calculated using the Mann-Whitney U test, the fifth band was left off because so few snores exhibited this formant. If a snore had more than one formant in a band the average frequency of those was used in the calculation. Bands 1 \& 3 showed significant differences between formants when comparing the standard deviation of simple snorers and OSA sufferers both in normal snores and post apneic snores of the OSA sufferers. The most distinct difference was between the standard deviation of simple snorers’ normal snores and post-apneic snores in band 1 which had a probability of 0.0006. 

Yadollahi et al used formant frequencies to distinguish between snores and breaths rather than simple snorers and OSA suffers so a direct comparison of method cannot be made however there is value in exploring the method used. Bands were used as in the Sola-Soler study but at different frequency ranges, [20−400]Hz, [270−840]Hz, [500−1380]Hz, [910−1920]Hz, [1680-2680]Hz, [2580-3770]Hz and [3590-5000]Hz these were found using K–means clustering. This is a method of partitioning data unsupervised, it uses an iterative method to find a predefined number of partitions, in this case seven, an error margin of $10^{−5}$ was used for the iterations. The process is sensitive to initial conditions so was repeated 20 times with different initial conditions, the one with the minimum error was selected. Table X (3) shows the student t-test p values, which show the first and third formants to be significant, with probability p = 0.003 and p = 0.0244 respectively. The F1 frequency of the breath sound was greater than that of the snore while the F3 frequencies with the converse. 

\subsubsection{Bispectral Analysis}
Bispectral analysis exploits the fact that the bispectrum reveals both amplitude and phase information about a spectrum, while also being calculated by convolution makes it the easiest polyspectra to compute. Polyspectra are Fourier Transforms of cumulants, for example the second order cumulant; autocorrelation Fourier Transforms to the Power Spectrum. The third-order cumulant C(m,n) = E{x(k)x(k+m)x(k+n)} transforms via a Double Discrete Fourier Transform (DDFT) to the bispectrum . Although the bispectrum is often plotted on a square it is symmetric and so only a triangular region is needed to completely describe it, this is defined by $0 \le \omega2 \le \omega1, \omega1 + \omega2 \le \pi$

Quadratic phase coupling occurs uniquely in second-order non-linear systems, and is where the phases add and subtract along with the frequency components. (QPC) causes peaks in the bispectrum triangular region, this shows energy is produced at frequency $\omega1 + \omega2$, a flat bispectrum at w1 and w2 suggests no activity and that it is not affected by QPC. 

Ng et al exploited the bispectrum shape in order to distinguish between nine OSA sufferers and seven simple snorers. Bispectral analysis has benefits over power spectrum because it reveals the non-Gaussian and non-linear behaviour, given the upper airway is non-linear this is a good plan. 

Noise was suppressed using a level-wavelet-dependent (LWD) thresholding scheme under undecimated discrete wavelet transform (UDWT) setting. Fast Fourier Transforms were used to estimate the bispectrum, this was plotted (figures X and X (4 \& 5) and inspected visually, the axies have been normalized with a frequency of 1 being 11025Hz. From visual inspection is appears the biggest peak for simple snorers are near the origin while for apneic snores are further away, this suggests a greater degree of phase coupling in apneic snores due to nonlinearities in the signals. When analysed peak position was reflected in the numbers with apneic peaks are higher frequencies than simple snore peaks, table X(1) shows the figures.

Clustered multiple comparison graphs were then produced, figure X(6), QPCs appear mainly at $f_1=f_2$ for simple snores whereas they appear mainly at $f_1 \not= f_2$ for apneic snores, however there is less self coupling. 77\% of simple snore are self coupled compared to 49\% of apneic snores. So the analysis shows there are three differences between simple snores and apneic snores, apneic snores have strong presence of nonlinear interaction, less self-coupling and QPC peaks appear and higher frequencies. 

\subsubsection{Wavelet Analysis}
Translation-Invariant Discrete Wavelet Transform(TIDWT) is a nonorthogonal, undecimated adaption of the more common Discrete Wavelet Transform (DWT) which is orthogonal and maximally decimated, and causes Gibbs like artefacts around discontinuities. There are two TIDWT methods:`a-trous and cycle-spinning. The `a-trous scheme removes the downsamplers and upsamplers of the DWT and upsamples the filter coefficients by a factor of $s^{j-1}$ in the $j^\text{th}$ level of the algorithm. As the output of each level contains the same number of samples as the input it is inherently redundant. 

Cycle spinning incorporates translation invariance by applying DWT to all circular shifts of the input vector and averaging over them. However due to periodic properties the number of shifts reduces to one. Both schemes offer better error properties than DWT due to redundant coefficients by about 10-20%. 

Ng et al used cycle spinning TIDWT to distinguish between 30 OSA sufferers and ten simple snorers. Ten snores of about 6 seconds were randomly chosen from each snorer, marked by polysomnogram technologists. Figure X(5) compares the energy approach used by Cavusoglu et al and the TIDWT approach used by Ng et al in this study. At the narrowest tolerance of 25ms the TIDWT approach detected 51\% of the snore segments correctly compared with 8\% for the energy approach, and at the largest tolerance of 125ms the TIDWT approach detected 98\% correctly, whereas the energy approach only managed 87\%, i.e. the TIDWT approach is at least 10\% more effective. 

\subsubsection{Multiscale Entropy Coefficients}
Entropy is used as a measure of complexness and the idea of using entropy for multiple scales was originally proposed by Zhang however this was based on Shannon’s definition of entropy that has some flaws so Costa et al proposed a new method using sample entropy (SampEn) a refinement on approximate entropy (ApEn) introduced by Pincus .

The method works by constructing consecutive coarse-grained time series of the original time series. These are created by averaging the data points within non-overlapping windows of increasing length. Each element of the coarse-grained series is calculated by the equation . An illustration of this is shown in figure X (1)

SampEn is then calculated for each coarse-grained time series and plotted as a function of the scale factor. SampEn quantifies the regularity and predictability of the time series, it acts as a regularity statistic, looking for patterns, it does this by looking at the probability that m consecutive data point that are similar will remain similar when one more data point is added, here similar means the distance between them is less than r. 

Roebuck and Clifford used MSE over 40 scales (1-40seconds) to distinguish between 50 simple snorer, 72 OSA sufferers and 24 normal subjects. They used 240minutes of recording from each subject, splitting the data between a training set and a testing set 70:30. Scales of 6seconds, 21seconds and 30second yielded a specificity of 92.5\% and a positive predictive value of 85.8\% in the training set and a specificity of 90.5\% and a positive predictive value of 83.5\% in the testing set.

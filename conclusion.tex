\chapter{Conclusion (Sachin)}
\label{ch:conclusion}
 
Obstructive Sleep Apneoa (OSA) is a serious and prevalent condition that affects many middle-aged people, and many of them remain undiagnosed, leading to poor sleep and a reduced quality of life. The current diagnosis solutions offered by the NHS are relatively expensive and inconvenient, and there is a need in the market for a way to check for OSA without having to spend a lot of time, money or effort. The easiest way to provide for this need would be through a smartphone application that would utilise machine learning to diagnose OSA from the user's audio sleep data.

After proving that there is a need and potential market for a mobile app that is able to provide a preliminary diagnosis of OSA, we moved on to first develop a method of diagnosis without using machine learning, to use as a backup and to familiarise ourselves with sleep data. We created a simple method of diagnosis in \verb!MATLAB!\textsuperscript{\textregistered} using power thresholds and variance analysis. This method was able to provide us with an output of 1s or 0s, highlighting different time periods when the user experiences apneoa. We then tested the simple method on a 7 minute audio file obtained from the internet, of a real OSA patient.

We then moved on to research several machine learning techniques that could be used in our app, and identified Support Vector Machines (SVM) and Hidden Markov Models (HMM) as being suitable in our case. We built the models in \verb!MATLAB!\textsuperscript{\textregistered} and ran tests using data obtained from PhysioNet to test the accuracy of both models. We presented the results and argued in Section \ref{sec:mlExperimentsSummary} that the HMM model would be better suited for the purpose of diagnosing OSA from temporal data. 
More importantly, what we were able to show was that if the appropriate model is used, OSA can be accurately diagnosed using only a mobile application on a smartphone and there is the potential to develop and launch it as a complementary service to the current NHS diagnosis and treatment service for OSA. This would save even more time and resources for doctors as the NHS approved questionnaires can be easily performed through the app even before meeting the GP.

In order to improve the accuracy of the diagnosis further, we need to improve both the quality and quantity of testing and training data. Firstly, using audio data (instead of the ECG data that we used in our initial experiments) would provide better applicability to the smartphone. Secondly, improving the quality, in terms of better annotations and multiple annotations by healthcare professionals to ensure accuracy of labelling would improve the accuracy of the model. Thirdly, simply providing more data for the model to estimate the parameters more accurately would, over time, make the app much better than it already is at identifying OSA in users. Lastly, the models we have developed can always be improved upon as well, as shown in Section \ref{sec:mlExperimentsSummary}.

We conclude with the assertion that the implementation and use of the mobile app we have developed would benefit the general public greatly, and also provide many benefits for healthcare professionals as well. We hope that further work will be done in this area to improve and refine the process, such that we finally find an app in the open market.
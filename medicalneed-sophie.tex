\section{Medical Need}

Medical need can essentially be split down into two areas: prevalence i.e. how many people suffer; and consequences, what people will die of and what their quality of life will be. There are a few other things that hint at the medical need and others’ opinions of that. 

The British Lung Foundation has prioritised OSA and created an OSA Charter calling on the UK governments to make OSA a national priority as well as encouraging investment in research. They held a three year campaign to raise awareness which ended in 2014 The campaign had two aims: to increase awareness both to the general public and health care professionals and to improve diagnosis. Part of the plan to improve diagnosis was to develop a national standard for diagnosis that would include a one stop shop so that the patient pathway would be reduced in length in order to reduce concerns about driving. A conference on OSA was held in February 2014 where doctors expressed desire to change the current system, due to increased need for doctor referral of patients to sleep centres. 

PhysioNet and Computers in Cardiology with funding from Margret and H.A. Rey Laboratory for Nonlinear Dynamics in Medicine set up a competition with two prizes of \$500 for whoever could classify ECG data, obtained minimally intrusively and inexpensively, into that from OSA sufferers and normal subjects, with the intension of using it as a screening tool. 

The Agency for Healthcare Research and Quality (AHRQ) felt that the diagnosis of sleep apnea was a sufficiently important public health issue that they commissioned a study on future research needs which includes a reference to the need for portable monitors, including limited-channel, low-cost portable devices.

\subsection{Prevalence}

Prevalence of OSA is hard to know due to the fact a high levels of cases go undiagnosed, however estimates range from 1 to 28% depending on severity and location. 

There may be an ethnicity element in prevalence, however few studies have been undertaken other than in western countries and therefore prevalence elsewhere is essentially unknown. However for some areas it is known, this means study on causes can be undertaken for example prevalence in Western Nations and Hong Kong is very similar however prevalence of risk factors is very different, there are high levels of obesity in the west but not in those studied from Hong Kong, so hypotheses have been produced on other risk factors including facial features being more prevalent in Hong Kong and some clinical observations support these. 

Gender has been shown to have an effect with estimates of 2 to 3 times great risk for men compared to women. However the reasons behind this are unclear. Hormones have been considered but administration of the female hormones oestrogen and progesterone to men does not appear to have an effect. Men show greater prevalences to many chronic diseases so this may be part of a greater trend and differences elsewhere have been shown to be linked to physical features, occupation, environment, attitude to health and risky behaviour. There are gender differences in upper airway shape, muscle activity, facial shape, and deposition of fat in the airway, however the few studies that have looked at this have yet to find a conclusive link. Whereas occupation, attitude to health and risky behaviour have not been studied while specificly looking at gender disparity in OSA. 

There are hypotheses proposing higher prevalence of OSA in pregnant mothers however few data to support this. Proposed mechanisms to cause this include excess weight gain and the effect of sleep deprivation on pharyngeal dilator muscle activity.

Although a positive trend between OSA prevalence and age appears to exist for mid life the same isn’t true for younger or older patients. OSA in children has similar consequence to that in adults and some of the pathophysiology (physical manifestation of a disease) is the same, however the etiology (causes) and associated morbidity (rate of incidence of a disease) can be very different, which means that it is generally studied independently from the adult form. 

In old age prevalence of OSA increases however this doesn’t necessarily mean that physiological changes associated with old age are causing OSA. If this was the case one would expect the prevalence rate to increase at the same rate as through middle age or at a higher rate. Figure X shows the Sleep Heart Health Study on prevalence with age which starts to flatten in the 60s which suggests age related prevalence tails off at this point. 

It is possible that older age OSA is actually distinct from that of middle age, several studies support this theory, as many of the key symptoms of middle aage OSA are not present in the old age version including, daytime sleepiness,obesity, decrease in cognitive function and hypertention. Snoring is also significantly less reported however this could be caused by increase in bed partner hearing loss and death. 

\subsection{Long Term Effects}

There is an association between OSA and secondary hypertension ( high blood pressure) independent of excess weight and other factors. This link is seen even in mild OSA, and given the prevalence of OSA could be having an impact on a significant proportion of those suffering with hypertension. However attempts to treat OSA in order to reduce hypertension have so far yielded unclear results. 

Hypertension is linked to cardiovascular and cerebrovascular disease and therefore given the link between OSA and hypertension, OSA will moderately contribute to the morbidity and mortality of these. There may also be direct links between OSA and cardiovascular disease however this has been less well studied. Whether treatment of OSA can improve cardiovascular disease has yet to be assessed. 

Daytime sleepiness is a primary feature of OSA and many studies have shown that treatment of OSA does reduce daytime sleepiness. Studies have found a significant association between snoring and daytime sleepiness. Snoring is a strong indicator of OSA, so the link between OSA and daytime sleepiness could be due to snoring, studies have shown the association between snoring and daytime sleepiness is independent of OSA. 

The effects of OSA on cognitive function is not fully understood, there are some population based studies which find weaker correlations than clinic based studies, this is probably due to the biased population who attend sleep laboratories. In one study OSA was significantly but weakly related to reduced psychomotor efficiency (a measure of coordination of fine motor control with sustained attention), this link was not explained by daytime sleepiness. In another study of self reported snorers a weak but significant association was found between OSA and neuropsychological function.

There is a thought that during OSA the restricted airflow causes a reduction in oxygen supply to the brain which in turn causes changes in the neurons of the hippocampus and the right frontal cortex. This atrophy has been shown using neuro-imaging to be irreversible even with CPCP and is seen in 25% of OSA cases. The effects are problems with mentally manipulating non-verbal information, and in working memory and executive functions. 

There is no specific quality of life measure for OSA although one is being developed, however the SF-36 a general health-related quality of life measure, a short form of the Medical Outcomes Study, is in use. A couple of studies have found a linear association between severity of OSA and decrements on the SF-36 scales, showing undiagnosed OSA has a similar affect on quality of life to other chronic disorders of similar severity, although another study showed a threshold effect as severity of OSA increased on a study of self-reported sleepiness or snoring, however a small sample size limits usability. 

Patients with OSA have a higher vehicle crash rate than the general populous, this has been shown by crash records, self-reports and performance on driving simulators. This is a significant issue because it puts the lives of everyone not just teh sufferers at risk. Studies undertaken in clinic will over estimate rate of crashes due to selection bias, however there are population studies studies looking at those with undiagnosed OSA which also show a strong correlation, especially among men. Self-reported sleepiness was not able to explain the crash rate. This is concerning because it means those at risk are not able to recognise it within themselves and are therefore unable to take precautions to reduce risk. 
